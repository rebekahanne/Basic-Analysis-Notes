%% 
%% This is file, `analysis_definitions.tex',
%% generated with the extract package.
%% 
%% Generated on :  2019/03/29,8:34
%% From source  :  "Analysis Notes".tex
%% Using options:  active,generate=analysis_definitions,extract-env={definition,algorithm}
%% 
\documentclass[11pt]{article}

%%%%%%%%%%%%%%%%%%%%%%%%%%%%%%%%%%%%%%%%%%%%%%%%%%%%%%%%%%%%%%%%%%%%%%%%%%%%%%%%%%

% Packages

% AMS
\usepackage{amsmath, amssymb, amsthm, amsbsy}
% Geometry
\usepackage{geometry}
% Colors
\usepackage[usenames,dvipsnames]{xcolor}
% Figures
\usepackage{float}
\usepackage{graphicx}
% Multi column lists
\usepackage{multicol}
% Subfigures
\usepackage{caption}
\usepackage{subcaption}
% Caligraphic
\usepackage{mathrsfs}
\usepackage{bbm}
% Bold
\usepackage{bm}
% algos
\usepackage[linesnumbered, lined, ruled]{algorithm2e}
% Spacing
\usepackage{setspace}
% Refs
\usepackage[colorlinks=true, citecolor=Blue, linkcolor=blue]{hyperref}
\newcommand\myshade{85}
\colorlet{mylinkcolor}{violet}
\colorlet{mycitecolor}{PineGreen}
\colorlet{myurlcolor}{Aquamarine}

\hypersetup{
  linkcolor  = mylinkcolor!\myshade!black,
  citecolor  = mycitecolor!\myshade!black,
  urlcolor   = myurlcolor!\myshade!black,
  colorlinks = true,
}
% Bibliography
\usepackage{filecontents}
\usepackage{natbib}
% Indent
\usepackage{indentfirst}
% Pretty lists
\usepackage{enumitem}
\setlist[enumerate]{itemsep=2pt,topsep=3pt}
\setlist[itemize]{itemsep=2pt,topsep=3pt}
\setlist[enumerate,1]{label=(\roman*)}

% Code
\usepackage{listings}

% Appendix
\usepackage[toc,page]{appendix}

% Math
\usepackage{mathtools}
\usepackage{xparse}

% Equation numbering
\numberwithin{equation}{section}

% Use more than one optional parameter in a new commands
\usepackage{xargs}
% Todo
\usepackage[colorinlistoftodos,prependcaption,textsize=normalsize]{todonotes}
\newcommandx{\unsure}[2][1=]{\todo[linecolor=red,backgroundcolor=red!25,bordercolor=red,#1]{#2}}
\newcommandx{\change}[2][1=]{\todo[linecolor=blue,backgroundcolor=blue!25,bordercolor=blue,#1]{#2}}
\newcommandx{\info}[2][1=]{\todo[linecolor=OliveGreen,backgroundcolor=OliveGreen!25,bordercolor=OliveGreen,#1]{#2}}
\newcommandx{\improvement}[2][1=]{\todo[linecolor=Plum,backgroundcolor=Plum!25,bordercolor=Plum,#1]{#2}}
\newcommandx{\thiswillnotshow}[2][1=]{\todo[disable,#1]{#2}}

% Framed theorems
\usepackage{mdframed}

%%%%%%%%%%%%%%%%%%%%%%%%%%%%%%%%%%%%%%%%%%%%%%%%%%%%%%%%%%%%%%%%%%%%%%%%%%%%%%%%%%

% Document Settings

% Figure path
\graphicspath{{./figures/}}
% Matrix columns
\setcounter{MaxMatrixCols}{10}
% So pages will break inside long equation environments
\allowdisplaybreaks
% Font
\usepackage{mathpazo}
\linespread{1.05}
%\usepackage{courier}
% Geometry
\geometry{left=1in,right=1in,top=1in,bottom=1in}
% Counters
\setcounter{tocdepth}{2}
\setcounter{secnumdepth}{3}

%%%%%%%%%%%%%%%%%%%%%%%%%%%%%%%%%%%%%%%%%%%%%%%%%%%%%%%%%%%%%%%%%%%%%%%%%%%%%%%%%%

% Colors

\definecolor{Tm}{rgb}{0,0,0.80}
\newcommand{\navy}[1]{\textcolor{MidnightBlue}{\bf #1}}

%%%%%%%%%%%%%%%%%%%%%%%%%%%%%%%%%%%%%%%%%%%%%%%%%%%%%%%%%%%%%%%%%%%%%%%%%%%%%%%%%%

% Environments

\theoremstyle{definition}
\newmdtheoremenv{theorem}{\color{ForestGreen}{\textbf{Theorem}}}[section]
% \newtheorem{theorem}{\color{ForestGreen}{\textbf{Theorem}}}[section]
\newtheorem{claim}{\color{ForestGreen}{\textbf{Claim}}}[section]
% \newtheorem{lemma}[theorem]{\color{ForestGreen}{\textbf{Lemma}}}
% \newtheorem{proposition}[theorem]{\color{ForestGreen}{\textbf{Proposition}}}
% \newtheorem{corollary}[theorem]{\color{ForestGreen}{\textbf{Corollary}}}
\newmdtheoremenv{lemma}[theorem]{\color{ForestGreen}{\textbf{Lemma}}}
\newmdtheoremenv{proposition}[theorem]{\color{ForestGreen}{\textbf{Proposition}}}
\newmdtheoremenv{corollary}[theorem]{\color{ForestGreen}{\textbf{Corollary}}}

\newtheorem{axiom}[theorem]{\color{ForestGreen}{\textbf{Axiom}}}
\newtheorem{conjecture}[theorem]{Conjecture}
\newtheorem{case}[theorem]{Case}
\newtheorem{conclusion}[theorem]{Conclusion}
\newtheorem{criterion}[theorem]{Criterion}
\newtheorem{notation}[theorem]{Notation}
\newtheorem{problem}[theorem]{Problem}

\theoremstyle{definition}
% \newtheorem{definition}{\color{MidnightBlue}{\textbf{Definition}}}[section]
\newmdtheoremenv{definition}{\color{MidnightBlue}{\textbf{Definition}}}[section]
\newtheorem{example}{\color{WildStrawberry}Example}[section]
\newtheorem{assumption}{Assumption}[section]
\newtheorem{condition}[assumption]{Condition}
\newtheorem*{solution}{\color{Goldenrod}Solution}
% \newenvironment{solution}[1][\proofname]{%
%   \proof[\bf \color{Goldenrod}Solution to #1]%
% }{\endproof}

\newtheorem{exercise}{\color{YellowOrange}Exercise}[section]

% Literature Summary Standards
\newtheorem*{motivation}{Motivation}
\newtheorem*{summary}{Summary}
\newtheorem*{remark}{Remark}
\newtheorem*{model}{Model}
\newtheorem*{tresults}{Theoretical Results}
\newtheorem*{eresults}{Empirical Results}

%%%%%%%%%%%%%%%%%%%%%%%%%%%%%%%%%%%%%%%%%%%%%%%%%%%%%%%%%%%%%%%%%%%%%%%%%%%%%%%%%%

% Math macros

% Math ``brackets''
\newcommand\parens[1]{\left( #1 \right)}
\newcommand\squares[1]{\left[ #1 \right]}
\newcommand\braces[1]{\left\{ #1 \right\}}
\newcommand\angles[1]{\left\langle #1 \right\rangle}
\newcommand\ceil[1]{\left\lceil #1 \right\rceil}
\newcommand\floor[1]{\left\lfloor #1 \right\rfloor}
\newcommand\abs[1]{\left| #1 \right|}
\newcommand\dabs[1]{\left\| #1 \right\|}
\newcommand\vect[1]{\mathbf{#1}}
\newcommand\closure[1]{\overline{#1}}
\newcommand\pset[1]{\mathcal{P}\left(#1\right)}
\newcommand\inv[1]{#1^{-1}}
\newcommand\norm[1]{\lVert#1\rVert}

% inner product
\providecommand{\inner}[1]{\left\langle{#1}\right\rangle}
% stochastic dominance
\newcommand{\lesd}{\preceq_{\textrm{SD}}}

% Set builder (use \Set ultimately and separate by ;)
\DeclarePairedDelimiterX{\set}[1]{\{}{\}}{\setargs{#1}}
\NewDocumentCommand{\setargs}{>{\SplitArgument{1}{;}}m}
{\setargsaux#1}
\NewDocumentCommand{\setargsaux}{mm}
{\IfNoValueTF{#2}{#1} {#1\nonscript\:\delimsize\vert\allowbreak\nonscript\:\mathopen{}#2}}%
\def\Set{\set*}%

% hat
\usepackage{scalerel,stackengine}
\stackMath
\newcommand\widehatt[1]{%
\savestack{\tmpbox}{\stretchto{%
  \scaleto{%
    \scalerel*[\widthof{\ensuremath{#1}}]{\kern-.6pt\bigwedge\kern-.6pt}%
    {\rule[-\textheight/2]{1ex}{\textheight}}%WIDTH-LIMITED BIG WEDGE
  }{\textheight}%
}{0.5ex}}%
\stackon[1pt]{#1}{\tmpbox}%
}

% Shortcut math
\newcommand{\ls}{\leqslant}
\newcommand{\gs}{\geqslant}
\def\ss{\subset}
\def\sse{\subseteq}
\def\nss{\not \ss}
\def\sps{\supset}
\def\pss{\subsetneq}
\def\prece{\preccurlyeq}
\def\condgap{\hspace{1cm}}
\def\eprec{\preceq}
% argmax and min
\newcommand{\argmax}{\operatornamewithlimits{argmax}}
\newcommand{\argmin}{\operatornamewithlimits{argmin}}
\newcommand{\es}{\emptyset}
% Implication and reverse implication
\def\imp{\Rightarrow}
\def\pmi{\Leftarrow}
% Integers up to number
\newcommand\intsfin[1]{\braces{1, \ldots, #1}}
% Logic
\def\bic{\Leftrightarrow}
% Bold and italic
\newcommand\boldit[1]{\textbf{\textit{#1}}}
% Misc math
\newcommand{\st}{\ensuremath{\ \mathrm{s.t.}\ }}
\newcommand{\setntn}[2]{ \{ #1 : #2 \} }
\newcommand{\cf}[1]{ \lstinline|#1| }
\newcommand{\fore}{\therefore \quad}
\newcommand{\tod}{\stackrel { d } {\to} }
\newcommand{\tow}{\stackrel { w } {\to} }
\newcommand{\toprob}{\stackrel { p } {\to} }
\newcommand{\toms}{\stackrel { ms } {\to} }
\newcommand{\eqdist}{\stackrel{d} {=} }
\newcommand{\iidsim}{\stackrel{\textrm{ {\sc iid }}} {\sim} }
\newcommand{\1}{\mathbbm 1}
\newcommand{\dee}{\,{\rm d}}
\newcommand{\given}{\, | \,}
\newcommand{\la}{\langle}
\newcommand{\ra}{\rangle}

% Shortcut greek
\def\a{\alpha}
\def\b{\beta}
\def\g{\gamma}
\def\D{\Delta}
\def\d{\delta}
\def\z{\zeta}
\def\k{\kappa}
\def\l{\lambda}
\def\n{\nu}
\def\r{\rho}
\def\s{\sigma}
\def\t{\tau}
\def\x{\xi}
\def\w{\omega}
\def\W{\Omega}
% Nice greek
\newcommand{\p}{\varphi}
\newcommand{\e}{\varepsilon}

% Shorcut vectors
\def\vx{\vect{x}}
\def\vy{\vect{y}}
\def\va{\vect{a}}
\def\vb{\vect{b}}

\newcommand{\CC}{\mathbb C}
\newcommand{\FF}{\mathbb F}
\newcommand{\RR}{\mathbb R}
\newcommand{\NN}{\mathbb N}
\newcommand{\PP}{\mathbbm P}
\newcommand{\EE}{\mathbbm E}
\newcommand{\TT}{\mathbbm T}
\newcommand{\VV}{\mathbbm V}
\newcommand{\QQ}{\mathbbm Q}
\newcommand{\WW}{\mathbbm W}
\newcommand{\ZZ}{\mathbbm Z}
\newcommand{\KK}{\mathbbm K}
\renewcommand{\SS}{\mathbbm S}

% Expectation/Probability
\newcommand{\ee}[1]{\mathbbm{E}[{#1}]}
\newcommand{\pp}[1]{\mathbbm{P}({#1})}

\newcommand{\GG}{\mathsf G}
\newcommand{\XX}{\mathsf X}
\renewcommand{\AA}{\mathsf A}
\newcommand{\YY}{\mathsf Y}
\newcommand{\ZZZ}{\mathsf Z}

\newcommand{\cC}{\mathscr C}
\newcommand{\iI}{\mathscr I}
\newcommand{\eE}{\mathscr E}
\newcommand{\fF}{\mathscr F}
\newcommand{\rR}{\mathscr R}
\newcommand{\sS}{\mathscr S}
\newcommand{\lL}{\mathscr L}
\newcommand{\cG}{\mathscr G}

\newcommand{\aA}{\mathcal A}
\newcommand{\pP}{\mathcal P}
\newcommand{\vV}{\mathcal V}
\newcommand{\dD}{\mathcal D}
\newcommand{\mM}{\mathcal M}
\newcommand{\oO}{\mathcal O}
\newcommand{\gG}{\mathcal G}
\newcommand{\hH}{\mathcal H}
\newcommand{\tT}{\mathcal T}
\newcommand{\bB}{\mathcal B}

% Common collections
\def\cA{\col{A}}
\def\cB{\col{B}}
\def\cC{\col{C}}
\def\cT{\col{T}}
\def\cU{\col{U}}

% Common closures
\def\clA{\closure{A}}
\def\clB{\closure{B}}
\def\clK{\closure{K}}

% operators
\DeclareMathOperator{\cl}{cl}
\DeclareMathOperator{\graph}{graph}
\DeclareMathOperator{\interior}{int}
\DeclareMathOperator{\Prob}{Prob}
\DeclareMathOperator{\determinant}{det}
\DeclareMathOperator{\trace}{trace}
\DeclareMathOperator{\sgn}{sgn}
\DeclareMathOperator{\Span}{span}
\DeclareMathOperator{\diag}{diag}
\DeclareMathOperator{\proj}{proj}
\DeclareMathOperator{\rank}{rank}
\DeclareMathOperator{\cov}{Cov}
\DeclareMathOperator{\corr}{Corr}
\DeclareMathOperator{\var}{Var}
\DeclareMathOperator{\mse}{mse}
\DeclareMathOperator{\se}{se}
\DeclareMathOperator{\row}{row}
\DeclareMathOperator{\col}{col}
\DeclareMathOperator{\range}{rng}
\DeclareMathOperator{\kernel}{ker}
\DeclareMathOperator{\dimension}{dim}
\DeclareMathOperator{\bias}{bias}
\DeclareMathOperator{\dom}{dom}
\DeclareMathOperator{\ran}{ran}
\DeclareMathOperator{\Int}{Int}
\DeclareMathOperator{\Cl}{Cl}


\begin{document}

\begin{definition}[Pointwise Convergence]
A sequence of functions $(f_n)_n$ \navy{converges pointwise} to a function $f$ on $E$ if for every $\e > 0$ and for all $x \in E$ there is an integer $N$ (which depends on $x$) such that for all $n \geq N$
\begin{equation*}
|f_n(x) - f(x)| < \e
\end{equation*}
\end{definition}

\begin{definition}[Uniform Convergence]
A sequence of functions $(f_n)_n$ \navy{converges uniformly} to a function $f$ on $E$ if for every $\e > 0$ there is an integer $N$ such that for all $n \geq N$ and for all $x \in E$ we have
\begin{equation*}
|f_n(x) - f(x)| < \e
\end{equation*}
\end{definition}

\begin{definition}[Uniformly cauchy]
A sequence of functions $(f_n)$ is \navy{uniformly cauchy} if $\forall \e > 0$ $\exists N \in \NN$ such that $\forall n,m \geq N$ and $\forall x \in X$, we have $|f_n(x) - f_m(y)| < \e$.
\end{definition}

\begin{definition}[Power Series]
A \navy{power series} is a function of the form
\begin{equation}
f(x) = \sum_{n=0}^\infty c_n x^n
\end{equation}
where $c_n \in \CC$ are complex coefficients.
\end{definition}

\begin{definition}[Radius of convergence]
To a power series we can associate a number $R \in [0, \infty]$ (thus $R$ is an \emph{extended} real number) called its \navy{radius of convergence} such that
\begin{enumerate}
\item $\sum_{n=0}^\infty c_n x^n$ converges for every $|x| < R$.
\item $\sum_{n=0}^\infty c_n x^n$ diverges for every $|x| > R$.
\end{enumerate}
\end{definition}

\begin{definition}[Open, open relative to]
Let $E \subset U \subset X$, where $X$ is a metric space.
\begin{enumerate}
\item $E$ is an \navy{open} subset of $X$ if for each point $p \in E$ there exists an $r > 0$ such that for all $q \in X$ for which $d(p,q) < r$ we have that $q \in E$.
\item $E$ is \navy{open relative to} $Y$ if for each point $p \in E$ there exists an $r > 0$ such that for all $q \in Y$ for which $d(p,q) < r$ we have that $q \in E$.
\end{enumerate}
\end{definition}

\begin{definition}[Continuous]
Suppose $X$ and $Y$ are metric spaces, $E \subset X$, $p \in E$, and $f$ maps $E$ into $Y$. Then $f$ is \navy{continuous} at $p$ if for every $\e > 0$ there there exists a $\d > 0$ such that for all $x \in E$ for which $d_X(x,p) < \d$, we have that $d_Y(f(x), f(p)) < \e$.
\end{definition}

\begin{definition}[Uniformly Continuous]
Let $f$ be a mapping of a metric space $X$ into a metric space $Y$. We say that $f$ is \navy{uniformly continuous} on $X$ if for every $\e > 0$ there exists a $\d > 0$ such that for all $p,q \in X$ for which $d_X(p,q) < \delta$, we have $d_Y(f(p),f(q)) < \e$.
\end{definition}

\begin{definition}[Dense] TFAE: $E$ is \navy{dense} in $X$ if
\begin{enumerate}
\item  Every point of $X$ is a limit point of $E$ or a point of $E$ (or both).
\item $\bar{E} = X$.
\item $\forall \e > 0$ and $\forall x \in X$ we have that $B(x,\e) \cap E \neq \emptyset$.
\end{enumerate}
\end{definition}

\begin{definition}[Open Cover]
Let $I$ be an arbitrary index set. A collection $(G_i)_{i \in I}$ of open sets $G_i \subset X$ is called an \navy{open cover} of $X$ if $X \subset \cup_{i \in I} G_{i}$.
\end{definition}

\begin{definition}[Compact]
$X$ is \navy{compact} if every open cover of $X$ contains a finite subcover. More explicitly, for every open cover $(G_i)_{i \in I}$ there exists $m \in \NN$ and $i_1, i_2, \ldots, i_m \in I$ such that $X \subset \cup_{j=1}^m G_{i_j}$.
\begin{remark}
This is also called the Heine-Borel property.
\end{remark}
\end{definition}

\begin{definition}[Compact subset]
A subset $A \subset X$ is called \navy{compact subset} if $(A,d\vert_{A\times A})$ is a compact metric space. $d\vert_{A\times A}$ is the restriction of $d$ to $A \times A$.
\end{definition}

\begin{definition}[Relatively compact or precompact]
A subset $A \ss X$ is called \navy{relatively compact} or \navy{precompact} if the closure $\bar{A} \ss X$ is compact.
\end{definition}

\begin{definition}[Sequentially compact]
A metric space $X$ is \navy{sequentially compact} if every sequence in $X$ has a convergent subsequence.
\begin{remark}
This is also called the Bolzano-Weierstrass property. Recall that the Bolzano-Weierstrass theorem states that every bounded sequence in $\RR^n$ has a convergent subsequence.
\end{remark}
\end{definition}

\begin{definition}[Bounded metric space]
A metric space $X$ is \navy{bounded} if it fits in a single fixed ball. More precisely, there exists some $x_0 \in X$ and $r > 0$ such that $X \subseteq B(x_0,r)$.
\end{definition}

\begin{definition}[Totally bounded]
A metric space $X$ is \navy{totally bounded} if for every $\e > 0$ there exist finitely many balls of radius $\e$ that cover $X$.
\end{definition}

\begin{definition}[Complete Orthonormal System]
An orthonormal system $(\phi_n)_n$ is called \navy{complete} if
\begin{equation}
\sum_{n=1}^\infty \abs{\angles{f,\phi_n}}^2 = \norm{f}^2_2 \quad \forall f \in pc([a,b])
\end{equation}
\end{definition}

\begin{definition}[Trigonometric Polynomial, Degree]
A \navy{trigonometric polynomial} is a function of the form
\begin{equation}
f(x) = \sum_{n=-N}^N c_n e^{2\pi i n x} \quad x \in \RR
\end{equation}
where $N \in \NN$ and $c_n \in \CC$. The largest $N$ for which either $c_N$ or $c_{-N}$ is non-zero is called the \navy{degree} of $f$.
\end{definition}

\begin{definition}[Partial sums]
For a 1-periodic function $f \in \text{pc}$ we define the \navy{partial sums}
\begin{equation}
S_Nf(x) = \sum_{n=-N}^N \hat{f}(n) e^{2\pi i nx}
\end{equation}
\end{definition}

\begin{definition}[Convolution]
For two 1-periodic functions $f,g \in \text{pc}$ we define their \navy{convolution} by
\begin{equation}
f * g(x) = \int_0^1 f(t)g(x-t)dt
\end{equation}
\end{definition}

\begin{definition}[Approximation of Unity]
A sequence of 1-periodic continuous functions $(k_n)_n$ is called \navy{approximation of unity} if for all 1-periodic continuous functions $f$ we have that $f * k_n$ converges uniformly to $f$ on $\RR$. That is
\begin{equation}
\sup_{x \in \RR} \abs{f * k_n(x) - f(x)} \to 0
\end{equation}
as $n \to \infty$
\end{definition}

\begin{definition}[Norm]
A map $\norm{\cdot}: X \to [0,\infty)$ is called a \navy{norm} if for all $x,y \in X$ and $\lambda \in \KK$ we have
\begin{enumerate}
\item $\norm{\lambda x} = |\lambda| \norm{x}$
\item $\norm{x + y} \leq \norm{x} + \norm{y}$
\item $\norm{x} = 0 \iff x = 0$
\end{enumerate}
\end{definition}

\begin{definition}[Normed vector space]
A $\KK$-vector space equipped with a norm is called a \navy{normed vector space}. On every normed vector space we have a natural metric space structure defined by
\begin{equation}
d(x,y) = \norm{x-y}
\end{equation}
\end{definition}

\begin{definition}[Banach space]
A complete normed vector space is called a \navy{Banach space}.
\end{definition}

\begin{definition}[Linear map]
Let $X,Y$ be normed vector spaces. A map $T: X \to Y$ is called \navy{linear} if
\begin{equation}
T(x + \lambda y) = Tx + \lambda T y
\end{equation}
for every $x,y \in X$, $\lambda \in \KK$.
\end{definition}

\begin{definition}[Bounded]
A linear map $T: X \to Y$ is called \navy{bounded} if there exists $C > 0$ such that
\begin{equation}
\norm{Tx}_Y \leq C \norm{x}_X \quad \forall x \in X
\end{equation}
\end{definition}

\begin{definition}[Supremum, Infimum]
Let $S$ be an ordered set, $E \subset S$, and $E$ be bounded above. Suppose there exists an $\alpha \in S$ with the following properties:
\begin{enumerate}
\item $\alpha$ is an upper bound of $E$.
\item $If \gamma < \alpha$ then $\gamma$ is not an upper bound of $E$.
\end{enumerate}
Then $\alpha$ is called the least upper bound of $E$ or \navy{supremum} of $E$ and we write $\alpha = \sup E$. Similarly, $\alpha$ is the greatest lower bound of $E$ or \navy{infimum} of $E$ if
\begin{enumerate}
\item $\alpha$ is a lower bound of $E$.
\item $If \beta > \alpha$ then $\beta$ is not a lower bound of $E$.
\end{enumerate}
and we write $\alpha = \inf E$.
\end{definition}

\begin{definition}[Limit Superior, Inferior]
Let $\parens{x_n}$ be a sequence of real numbers.

\begin{enumerate}
\item The \navy{limit superior} of the sequence is defined by
\begin{equation}
\limsup\limits_{n \to \infty} x_n = \lim_{n \to \infty}\parens{\sup_{m \geq n} x_m}
\end{equation}
or
\begin{equation}
\limsup\limits_{n \to \infty} x_n = \inf_{n \geq 0} \parens{\sup_{m \geq n} x_m} = \inf\Set{\sup \Set{x_m; m \geq n}; n \geq 0}
\end{equation}

\textbf{Alternatively}, the limit superior of the sequence is the smallest $b \in \RR$ such that $\forall \e > 0$ $\exists N$ such that $x_n < b + \e$ $\forall n > N$. Thus, any number larger than the limit superior is an upper bound for the sequence after a finite number of terms (hence, only a finite number of elements are greater that $b + \e$).

\textbf{Alternatively}, the limit superior of the sequence is the supremum of the set of subsequential limits.

\item The \navy{limit inferior} of the sequence is defined by
\begin{equation}
\liminf\limits_{n \to \infty} x_n = \lim_{n \to \infty}\parens{\inf_{m \geq n} x_m}
\end{equation}
or
\begin{equation}
\liminf\limits_{n \to \infty} x_n = \sup_{n \geq 0} \parens{\inf_{m \geq n} x_m} = \sup\Set{\inf \Set{x_m; m \geq n}; n \geq 0}
\end{equation}
\end{enumerate}

\end{definition}

\begin{definition}[Limit point]
A point $p$ is a \navy{limit point} of the set $E$ if every neighborhood of $p$ contains a point $q \neq p$ such that $q \in E$.
\end{definition}

\begin{definition}[Closed]
$E$ is \navy{closed} if every limit point of $E$ is a point of $E$.
\end{definition}

\begin{definition}[Interior]
A point $p$ is an \navy{interior point} of $E$ if there is a neighborhood $N$ of $p$ such that $N \subset E$.
\end{definition}

\begin{definition}[Open]
$E$ is \navy{open} if every point of $E$ is an interior point of $E$.
\end{definition}

\begin{definition}[Bounded]
$E$ is \navy{bounded} if there is a real number $M$ and a point $q \in X$ such that $d(p,q) < M$ for all $p \in E$.
\end{definition}

\begin{definition}[Separated, Connected]
Two subsets $A$ and $B$ of a metric space $X$ are said to be \navy{separated} if both $A \cap \bar{B}$ and $\bar{A} \cap B$ are empty (i.e., if no point of $A$ lies in the closure of $B$ and no point of $B$ lies in the closure of $A$). A set $E \subset X$ is said to be \navy{connected} if $E$ is not a union of two nonempty separated sets.
\end{definition}

\begin{definition}[Convergent Sequence]
A sequence $(p_n)$ in a metric space $X$ is said to \navy{converge} if there is a point $p \in X$ such that for every $\e > 0$ there is an integer $N$ such that $n \geq N$ implies $d(p_n,p) < \e$.
\end{definition}

\begin{definition}[Subsequence, Subsequential Limit]
Given a sequence $(p_n)$, consider a sequence $(n_k)$ of positive integers such that $n_1 < n_2 < n_3 < \cdots $. Then the sequence $(p_{n_i})$ is called a \navy{subsequence} of $(p_n)$. If $(p_{n_i})$ converges, its limit is called a \navy{subsequential limit} of $(p_n)$.
\end{definition}

\begin{definition}[Cauchy Sequence]
A sequence $(p_n)$ in a metric space $X$ is said to be a \navy{Cauchy Sequence} if for every $\e > 0$ there is an integer $N$ such that $d(p_n, p_m) < \e$ if $n \geq N$ and $m \geq N$.
\end{definition}

\begin{definition}[Diameter]
Let $E$ be a nonempty subset of a metric space $X$, and let $S$ be the set of all real numbers of the form $d(p,q)$ with $p \in E$ and $q \in E$. The sup of $S$ is called the \navy{diameter} of $E$.
\end{definition}

\begin{definition}[Complete]
A metric space in which every Cauchy sequence converges is \navy{complete}.
\end{definition}

\begin{definition}[Monotonically increasing, decreasing]
A sequence $(s_n)$ of real numbers is said to be
\begin{enumerate}
\item \navy{Monotonically increasing} if $s_n \leq s_{n_1}$ for all $n$.
\item \navy{Monotonically decreasing} if $s_n \geq s_{n_1}$ for all $n$.
\end{enumerate}
\end{definition}

\begin{definition}[Convergent Series]
Let $\sum_{n=1}^\infty a_n$ be an infinite series. Define $s_n = \sum_{k=1}^n a_n$ to be the $n$th partial sum of the series. If the sequence of partial sums $\Set{s_n}$ converges to $s$, we say the series $\navy{converges}$.
\end{definition}

\begin{definition}[Limit]
Let $X$ and $Y$ be metric spaces, $E \subset X$, $f$ map $E$ into $Y$, and $p$ be a limit point of $E$. We write $f(x) \to q$ as $x \to p$ or $\navy{\lim_{x\to p} f(x) = q}$ if there is a point $q \in Y$ such that for every $\e > 0$ there exists a $\d > 0$ such that for all $x \in E$ for which $0 < d_X(x,p) < \d$, we have $d_Y(f(x),q) < \e$.
\end{definition}

\begin{definition}[Continuous]
Suppose $X$ and $Y$ are metric spaces, $E \subset X$, $p \in E$, and $f$ maps $E$ into $Y$. Then $f$ is \navy{continuous} at $p$ if for every $\e > 0$ there there exists a $\d > 0$ such that for all $x \in E$ for which $d_X(x,p) < \d$, we have that $d_Y(f(x), f(p)) < \e$.
\end{definition}

\begin{definition}[Uniformly Continuous]
Let $f$ be a mapping of a metric space $X$ into a metric space $Y$. We say that $f$ is \navy{uniformly continuous} on $X$ if for every $\e > 0$ there exists a $\d > 0$ such that for all $p,q \in X$ for which $d_X(p,q) < \delta$, we have $d_Y(f(p),f(q)) < \e$.
\end{definition}

\begin{definition}[Differentiable, Derivative]
Let $f$ be defined (and real-valued) on $[a,b]$. For any $x \in [a,b]$ define
\begin{equation}
f'(x) = \lim_{t \to x} \frac{f(t) - f(x)}{t - x}
\end{equation}
provided this limit exists. If $f'$ is defined at a point $x$, we say that $f$ is \navy{differentiable} at $x$. $f'$ is called the \navy{derivative} of $f$.
\end{definition}

\end{document}
