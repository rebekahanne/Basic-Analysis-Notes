%% 
%% This is file, `analysis_theorems.tex',
%% generated with the extract package.
%% 
%% Generated on :  2019/01/01,11:57
%% From source  :  "Analysis Notes".tex
%% Using options:  active,generate=analysis_theorems,extract-env={theorem,corollary,claim}
%% 
\documentclass[12pt]{article}
\usepackage{amsthm}
\usepackage{amsfonts, amsmath}
\usepackage[dvipsnames]{xcolor}

\theoremstyle{definition}
\definecolor{Tm}{rgb}{0,0,0.80}
\newcommand{\navy}[1]{\textcolor{MidnightBlue}{\bf #1}}
\newtheorem{definition}{\color{MidnightBlue}{\textbf{Definition}}}
\newtheorem{theorem}{\color{ForestGreen}{\textbf{Theorem}}}
\newtheorem{corollary}{\color{ForestGreen}{\textbf{Corollary}}}
\usepackage{mathrsfs}

% algos
\usepackage[linesnumbered, lined, ruled]{algorithm2e}

   \usepackage{bm}
   \usepackage{bbm}

\newcommand{\ls}{\leqslant}
\newcommand{\gs}{\geqslant}
\newcommand{\vp}{\varphi}
\newcommand{\ve}{\varepsilon}

\newcommand{\e}{\epsilon}
\newcommand{\dd}{\delta}
\newcommand{\D}{\Delta}

\newcommand{\R}{\mathbb{R}}
\newcommand{\Z}{\mathbb{Z}}
\newcommand{\U}{\mathcal{U}}
\newcommand{\T}{\mathbb{R}}
\usepackage{mathtools}
\DeclarePairedDelimiterX{\inp}[2]{\langle}{\rangle}{#1, #2}
\newcommand{\norm}[1]{\lVert#1\rVert}
\newcommand{\x}{\bm{x}}
\newcommand{\xib}{\bm{\xi}}

\newcommand{\CC}{\mathbbm C}
\newcommand{\FF}{\mathbbm F}
\newcommand{\RR}{\mathbbm R}
\newcommand{\NN}{\mathbbm N}
\newcommand{\PP}{\mathbbm P}
\newcommand{\EE}{\mathbbm E}
\newcommand{\TT}{\mathbbm T}
\newcommand{\VV}{\mathbbm V}
\newcommand{\QQ}{\mathbbm Q}
\newcommand{\WW}{\mathbbm W}
\newcommand{\ZZ}{\mathbbm Z}
\renewcommand{\SS}{\mathbbm S}

\newcommand{\GG}{\mathsf G}
\newcommand{\XX}{\mathsf X}
\renewcommand{\AA}{\mathsf A}
\newcommand{\YY}{\mathsf Y}
\newcommand{\ZZZ}{\mathsf Z}

\newcommand{\aA}{\mathscr A}
\newcommand{\cC}{\mathscr C}
\newcommand{\iI}{\mathscr I}
\newcommand{\eE}{\mathscr E}
\newcommand{\fF}{\mathscr F}
\newcommand{\rR}{\mathscr R}
\newcommand{\sS}{\mathscr S}
\newcommand{\lL}{\mathscr L}

\newcommand{\cG}{\mathscr G}

\newcommand{\pP}{\mathcal P}
\newcommand{\vV}{\mathcal V}
\newcommand{\dD}{\mathcal D}
\newcommand{\mM}{\mathcal M}
\newcommand{\oO}{\mathcal O}
\newcommand{\gG}{\mathcal G}
\newcommand{\hH}{\mathcal H}
\newcommand{\tT}{\mathcal T}
\newcommand{\bB}{\mathcal B}

% operators
\DeclareMathOperator{\cl}{cl}
\DeclareMathOperator{\graph}{graph}
\DeclareMathOperator{\interior}{int}
\DeclareMathOperator{\Prob}{Prob}
\DeclareMathOperator{\determinant}{det}
\DeclareMathOperator{\trace}{trace}
\DeclareMathOperator{\sgn}{sgn}
\DeclareMathOperator{\Span}{span}
\DeclareMathOperator{\diag}{diag}
\DeclareMathOperator{\proj}{proj}
\DeclareMathOperator{\rank}{rank}
\DeclareMathOperator{\cov}{Cov}
\DeclareMathOperator{\corr}{Corr}
\DeclareMathOperator{\var}{Var}
\DeclareMathOperator{\mse}{mse}
\DeclareMathOperator{\se}{se}
\DeclareMathOperator{\row}{row}
%\DeclareMathOperator{\col}{col}
\DeclareMathOperator{\range}{rng}
\DeclareMathOperator{\kernel}{ker}
\DeclareMathOperator{\dimension}{dim}
\DeclareMathOperator{\bias}{bias}
\DeclareMathOperator{\colspace}{colspace}

% argmax and min
\newcommand{\argmax}{\operatornamewithlimits{argmax}}
\newcommand{\argmin}{\operatornamewithlimits{argmin}}

% sets
\newcommand{\Int}{\text{Int }}
\newcommand{\Cl}{\text{Cl }}

\begin{document}

\begin{theorem}[Sequences in compact metric spaces have a convergent subsequence]
If $(p_n)$ is a sequence in a compact metric space $X$, then some subsequence of $(p_n)$ converges to a point of $X$.
\end{theorem}

\begin{theorem}[Bolzano-Weierstrass]
Every bounded sequence in $\RR^k$ has a convergent subsequence.
\end{theorem}

\begin{theorem}[Facts about Cauchy sequences]
We have that
\begin{enumerate}
\item In any metric space $X$, every convergent sequence is a Cauchy sequence.
\item If $X$ is a compact metric space and if $(p_n)$ is a Cauchy sequence in $X$, then $(p_n)$ converges to some point of $X$.
\item In $\R^k$, every Cauchy sequence converges.
\end{enumerate}
\end{theorem}

\begin{theorem}[Convergence of monotonic sequences]
Let $(s_n)$ be a monotonic sequence. Then $(s_n)$ converges if and only if it is bounded.
\end{theorem}

\begin{theorem}[Continuous mappings on compact sets are uniformly continuous.]
Let $X,Y$ be metric spaces and assume $X$ is compact. If $f: X \to Y$ is continuous, then it is uniformly continuous.
\end{theorem}

\begin{theorem}[The image of a continuous function which maps from a compact set is compact]
Let $X,Y$ be metric spaces and assume $X$ is compact. If $f: X \to Y$ is continuous, then $f(X) \subset Y$ is compact.
\end{theorem}

\begin{theorem}[Mean Value Theorem]
If $f$ is a real continuous function on $[a,b]$ which is differentiable on $(a,b)$, then there is a point $x \in (a,b)$ at which
\begin{equation}
f(b) - f(a) = (b - a) f'(x)
\end{equation}
\end{theorem}

\begin{theorem}[Banach Fixed-Point Theorem]
If $X$ is a complete metric space, and if $\vp$ is a contraction of $X$ into $X$, then there exists a unique $x \in X$ such that $\vp(x) = x$.
\end{theorem}

\end{document}
